\documentclass[12pt,aspectratio=169]{beamer}
\usetheme{default}
\usecolortheme{dolphin}
\usefonttheme{structurebold}
\setbeamertemplate{footline}[frame number]

\title{Network 07}
\author{@aoirint}
% \institute{}
\date{2020/07/30}

\begin{document}

% 01
\frame{\maketitle}

% 02
\begin{frame}{テキスト}

  \begin{minipage}{0.58\textwidth}
    \begin{itemize}
      \item ネットワークがよくわかる教科書
      \begin{itemize}
        \item 著・福永勇二
        \item 刊・SB Creative
      \end{itemize}
      \item 今回の内容
        \begin{itemize}
          \item Chapter 2 Section 16
          \item Chapter 3 Section 1の途中まで
        \end{itemize}

    \end{itemize}

  \end{minipage}
  \hfill
  \begin{minipage}{0.38\textwidth}
    \vspace{-1\baselineskip}
    \begin{figure}[h]
      \centering
      \includegraphics[width=4cm,bb=0 0 420 596]{../02/figures/networkbook.jpg}
      \label{fig:networkbook}
      \caption{テキスト}
    \end{figure}
  \end{minipage}

  \begin{itemize}
    \item 書影
    \begin{itemize}
      \item { \small \url{https://www.sbcr.jp/product/4797393804/} }
    \end{itemize}
  \end{itemize}

\end{frame}


\begin{frame}{今回の内容}

  \begin{itemize}
    \item IPv6の概要
    \item 有線LANの基礎知識
      \begin{itemize}
        \item イーサネットの仕様と種類
      \end{itemize}

  \end{itemize}

\end{frame}

\begin{frame}{IPv6:IPv4の状況}

  \begin{itemize}
    \item IPv4
    \begin{itemize}
      \item 32bit(\( 2^{32} \)個)のIPアドレス
      \item IANAの割り当てるIPアドレスブロックは2011年2月に枯渇 \cite{jpnic:ip枯渇}
      \item APNIC/JPNICの割り当てるIPアドレスブロックは2011年4月に枯渇 \cite{jpnic:ip枯渇}
      % APNICとJPNICはIPアドレスの在庫を共有している
    \end{itemize}

    \centering
    \begin{figure}
      \centering
      \includegraphics[height=4cm,bb=0 0 950 480]{figures/ipv4stat.jpg}
      \label{fig:ipv4stat}
      \caption{2017年1月5日の在庫状況(\cite{jpnic:ip枯渇}より引用)}
    \end{figure}

  \end{itemize}

\end{frame}

\begin{frame}{IPv6:IPv4の状況}

  % 固定IPアドレスサービスではない(結構頻繁にIPアドレスが変わる)
  % 割り当てられるのはIPv4のグローバルアドレス
  \centering
  \begin{figure}
    \centering
    \includegraphics[height=6cm,bb=0 0 1009 570]{figures/wimaxglobal1.png}
    \label{fig:wimaxglobal1}
    \caption{WiMAXの有料オプション(\cite{wimax:napt}より引用)}
  \end{figure}

\end{frame}

\begin{frame}{IPv6:IPv4の状況}

  \centering
  \begin{figure}
    \centering
    \includegraphics[height=6cm,bb=0 0 934 481]{figures/wimaxglobal2.png}
    \label{fig:wimaxglobal2}
    \caption{WiMAXのNAPT(\cite{wimax:napt}より引用)}
  \end{figure}

\end{frame}

\begin{frame}{IPv6:IPv4の状況}

  \begin{itemize}
    \item IPv4ネットワークの今後\cite{jpnic:ip枯渇}
    \begin{itemize}
      \item 分配済みのIPアドレスを効率的に利用
      \item NAT/NAPTを利用しグローバルアドレスの分配を受けずに機器を追加
      \item IPv6へ移行
    \end{itemize}
  \end{itemize}

\end{frame}

\begin{frame}{IPv6:機器の更新}

  \begin{itemize}
    \item 既存のIPv4との間に互換性はない % 同時に運用することはできる
    \item L3以上の機器を交換・更新する必要がある
    \begin{itemize}
      \item コンピュータ
      \item ルータ
      \item ISPのルータ
      \item サーバ事業者のサーバ
    \end{itemize}
    \item L2以下の機器を取り替える必要はない
    \begin{itemize}
      \item LANケーブル
      \item ハブ・スイッチ
    \end{itemize}
  \end{itemize}

\end{frame}

\begin{frame}{IPv6:IPv6で通信するには}

  \begin{itemize}
    \item IPv6で通信するには
    \begin{itemize}
      % IPv6が出てきたのは結構前なので、それなりに最近の機器なら対応自体はしていると思われる
      \item 手元の機器(コンピュータやルータ)の設定
      \item ISPにIPv6でのインターネットアクセスを申し込む
    \end{itemize}
    \item IPv4とIPv6の混在した状況 % で通信する機器
  \end{itemize}

  \centering
  \begin{figure}
    \centering
    \includegraphics[height=2cm,bb=0 0 186 186]{figures/kamenetqr.png}
    \label{fig:kamenetqr}
    \caption{\texttt{ http://www.kame.net/ }}
  \end{figure}

\end{frame}

\begin{frame}{IPv6:IPv4のパケット構造}

  \centering
  \begin{figure}
    \centering
    \includegraphics[height=6cm,bb=0 0 1404 772]{figures/text_fig2_20.jpg}
    \label{fig:text_fig2_20}
    \caption{IPv4のパケット構造(テキストp.55から引用)}
  \end{figure}

\end{frame}

\begin{frame}{IPv6:IPv6のパケット構造}

  \centering
  \begin{figure}
    \centering
    \includegraphics[height=6cm,bb=0 0 1400 1172]{figures/text_fig2_62.jpg}
    \label{fig:text_fig2_62}
    \caption{IPv6のパケット構造(テキストp.98から引用)}
  \end{figure}

\end{frame}

\begin{frame}{IPv6:IPv6のパケット構造}

  \begin{itemize}
    \item 基本的なヘッダ
    \begin{itemize}
      \item 40バイト(固定長)
      \item 拡張ヘッダはペイロード扱い
      \item フィールド数は減った
    \end{itemize}
    \item ヘッダチェックサムの削除
    \begin{itemize}
      \item 上位プロトコルで検査
    \end{itemize}
    \item フラグメント関連は拡張ヘッダへ
    % ID, フラグメントオフセット
    \begin{itemize}
      \item 経路上でのパケット分割の廃止(送信元による分割除く)
    \end{itemize}
  \end{itemize}

\end{frame}

\begin{frame}{IPv6:IPv6のパケット構造}

  \begin{minipage}{0.48\textwidth}
    \small
    \begin{itemize}
      \item 次ヘッダフィールド
      \begin{itemize}
        \item 次のヘッダを指定
        \item 拡張ヘッダのヘッダ番号
        \item 上位プロトコルのプロトコル番号
      \end{itemize}
    \end{itemize}

  \end{minipage}
  \hfill
  \begin{minipage}{0.48\textwidth}
    \vspace{-1\baselineskip}
    \centering
    \begin{figure}
      \centering
      \includegraphics[width=7.5cm,bb=0 0 1420 474]{figures/text_tbl2_9.jpg}
      \label{fig:text_tbl2_9}
      \caption{IPv6の拡張ヘッダ(テキストp.99から引用)}
    \end{figure}
  \end{minipage}

\end{frame}

\begin{frame}{IPv6:IPv6アドレスの表記}

  \begin{itemize}
    \item 4ケタの16進数を8つ並べる
    \begin{itemize}
      \item 2001:0db8:0000:0000:0200:5eff:fe00:5301
    \end{itemize}
    \item 簡潔に書くための省略表記
    \begin{itemize}
      \item 2001:db8::200:5eff:fe00:5301
    \end{itemize}
    \item ::0や::1も有効なアドレス
    \begin{itemize}
      \item 0000:0000:0000:0000:0000:0000:0000:0000:未指定アドレス
      \item 0000:0000:0000:0000:0000:0000:0000:0001:ループバックアドレス
    \end{itemize}
  \end{itemize}

\end{frame}

\begin{frame}{IPv6:IPv6アドレスの省略表記}

  \centering
  \begin{figure}
    \centering
    \includegraphics[height=6cm,bb=0 0 1262 1292]{figures/text_fig2_63.jpg}
    \label{fig:text_fig2_63}
    \caption{IPv6アドレスの省略表記(テキストp.100から引用)}
  \end{figure}

\end{frame}

\begin{frame}{IPv6:IPv6アドレスの種類と形式}

  \begin{itemize}
    % ユニキャスト:相手が1つ
    \item グローバルユニキャストアドレス
    \begin{itemize}
      \item インターネット上で唯一
    \end{itemize}
    \item リンクローカルユニキャストアドレス
    \begin{itemize}
      \item ルータで中継されない
      \item 特定のイーサネット内でのみ有効
      \item 機器をネットワークに接続すると自動で割り振られる(DHCP不要)
    \end{itemize}
  \end{itemize}

\end{frame}

\begin{frame}{IPv6:IPv6アドレスの種類と形式}

  \begin{minipage}{0.48\textwidth}
    \small
    \begin{itemize}
      \item ネットワークプレフィックス:64bit
      \begin{itemize}
        \item IPv4のネットワーク部に相当
        \item ネットワークを表す
      \end{itemize}
      \item インタフェースID:64bit
      \begin{itemize}
        \item IPv4のホスト部に相当
        \item 端末や機器を特定する役割
        \item 固定長で運用(IPv4は可変:クラス)
      \end{itemize}
    \end{itemize}

  \end{minipage}
  \hfill
  \begin{minipage}{0.48\textwidth}
    \vspace{-1\baselineskip}
    \centering
    \begin{figure}
      \centering
      \includegraphics[width=6cm,bb=0 0 1394 968]{figures/text_fig2_64.jpg}
      \label{fig:text_fig2_64}
      \caption{IPv6アドレスの種類と形式(テキストp.101から引用)}
    \end{figure}
  \end{minipage}

\end{frame}

\begin{frame}{IPv6:ネットワークプレフィクス}

  \begin{minipage}{0.48\textwidth}
    \small
    \begin{itemize}
      \item ネットワークプレフィックス:64bit
      \begin{itemize}
        \item ルーティングプレフィックス(48-64bit)
        \item サブネットID(16-0bit)
      \end{itemize}
      \item ルーティングプレフィクス
      \begin{itemize}
        \item 割り当てを受けた値を使用
      \end{itemize}
      \item サブネットID
      \begin{itemize}
        \item サブネットを設けるときに使用
      \end{itemize}
    \end{itemize}

  \end{minipage}
  \hfill
  \begin{minipage}{0.48\textwidth}
    \vspace{-1\baselineskip}
    \centering
    \begin{figure}
      \centering
      \includegraphics[width=6cm,bb=0 0 1394 968]{figures/text_fig2_64.jpg}
      \label{fig:text_fig2_64}
      \caption{IPv6アドレスの種類と形式(テキストp.101から引用)}
    \end{figure}
  \end{minipage}

\end{frame}

\begin{frame}{IPv6:IPv6アドレスの種類と形式}

  \begin{minipage}{0.48\textwidth}
    \small
    \begin{itemize}
      \item リンクローカルユニキャストアドレスのルーティングプレフィックス
      \begin{itemize}
        \item 1111 1110 10 (0が54個)
        \item fe80:0000:0000:0000
      \end{itemize}
      \item IPv6では1つのNICに複数のIPv6アドレスが割り当てられるのが普通
      \begin{itemize}
        \item グローバルユニキャストアドレス
        \item リンクローカルアドレス
        \item 端末が適切なものを選択して使う
      \end{itemize}
    \end{itemize}

  \end{minipage}
  \hfill
  \begin{minipage}{0.48\textwidth}
    \vspace{-1\baselineskip}
    \centering
    \begin{figure}
      \centering
      \includegraphics[width=6cm,bb=0 0 1394 968]{figures/text_fig2_64.jpg}
      \label{fig:text_fig2_64}
      \caption{IPv6アドレスの種類と形式(テキストp.101から引用)}
    \end{figure}
  \end{minipage}

\end{frame}

\begin{frame}{IPv6:IPv4の重要な機能との対応}

  \begin{itemize}
    \item DHCPサーバが不要
    \begin{itemize}
      \item IPアドレスの自動割り当てを行うサーバ % 家庭の場合プライベートIPだがルータなどに実装、実際にはグローバルIPの割り当てをするDHCPサーバがあると思われる
      \item IPv6ではプレフィックスをルータが配信、機器がインタフェースIDを生成してIPv6アドレスを自動生成できる
      \item DHCPサーバを使うこともできる
    \end{itemize}
    \item NAT/NAPTが不要
    \begin{itemize}
      \item 内部の機器を隠すため使うこともできる
      \item プライベートIPアドレス
      \begin{itemize}
        \item ユニークローカルユニキャストアドレス
        \item このアドレスでインターネットと通信はできない
      \end{itemize}
    \end{itemize}
    \item ARP
    \begin{itemize}
      \item IPv6のICMPで実装、IPv6用の"ARP"はない
    \end{itemize}
  \end{itemize}

\end{frame}

\begin{frame}{イーサネット}

  \begin{itemize}
    \item 有線で接続するネットワークの標準規格
    \begin{itemize}
      \item 電気・光信号の形式
      \item コネクタ形状
      \item 対向する機器の通信方式など
    \end{itemize}
    \item 過去のイーサネット以外の有線ネットワーク規格
    \begin{itemize}
      \item トークンリング
      \item FDDI
      \item ATMなど
    \end{itemize}
    \item 現在はほぼすべてイーサネットに置き換えられた
  \end{itemize}

\end{frame}

\begin{frame}{イーサネット:ネットワークインタフェースカード}

  \begin{itemize}
    \item ネットワークインタフェースカード(NIC)
    \item コンピュータやネットワーク機器に組み込まれる通信用コンポーネント
    \begin{itemize}
      \item 一般的にはマザーボードに内蔵
      \item USB NIC(Nintendo Switchなど)
      \item PCI Expressの拡張ボード(デスクトップPCのNICを増設)など
      \item 1ポートだけでなく8ポート増設するようなものもある
      % もっと古いところで行くとゲームキューブのブロードバンドアダプタ、Wiiも有線は増設するNICかな
    \end{itemize}
    \item LANケーブルや光ケーブルを接続するポート
    \begin{itemize}
      \item 無線LANカード、Bluetoothなど:無線(アンテナ)
    \end{itemize}
  \end{itemize}

\end{frame}

\begin{frame}{イーサネット:NIC拡張}

  \begin{figure}
    \centering
    \includegraphics[height=4cm,bb=0 0 1417 452]{figures/text_fig3_1.jpg}
    \label{fig:nic}
    \caption{USB NICとPCI Express NIC(テキストp.107から引用)}
  \end{figure}

\end{frame}

\begin{frame}{イーサネット:NIC拡張}

  \begin{minipage}{0.48\textwidth}
    \begin{figure}
      \centering
      \includegraphics[width=6cm,bb=0 0 1280 720]{figures/switch_usb_nic.jpg}
      \label{fig:swnic}
      \caption{USB NICとNintendo Switch \cite{ninten:sw}}
    \end{figure}
  \end{minipage}
  \hfill
  \begin{minipage}{0.48\textwidth}
    \vspace{-1\baselineskip}
    \centering
    \begin{figure}
      \centering
      \includegraphics[height=6cm,bb=0 0 930 978]{figures/gc_bba_nic.jpg}
      \label{fig:gcnic}
      \caption{GCブロードバンドアダプタ \cite{ninten:gc}}
    \end{figure}
  \end{minipage}

\end{frame}

\begin{frame}{イーサネット:コネクタ形状}

    \begin{minipage}{0.48\textwidth}
      \small
      \begin{itemize}
        \item RJ-45コネクタ
        \begin{itemize}
          \item 一般的なLANケーブル
          \item 電話用モジュラープラグに似ている
          \item 8端子
        \end{itemize}
        \item TERAコネクタ
        \begin{itemize}
          \item データセンター向け\cite{jeitacat8}高カテゴリLANケーブル
        \end{itemize}
        \item SCコネクタ:ロックなし
        \begin{itemize}
          \item 送受信用の2つのコネクタが連なった形
        \end{itemize}
        \item LCコネクタ:小型化、ロック付き
      \end{itemize}
    \end{minipage}
    \hfill
    \begin{minipage}{0.48\textwidth}
      \begin{figure}
        \centering
        \includegraphics[width=6cm,bb=0 0 1416 784]{figures/text_fig3_2.jpg}
        \label{fig:connector}
        \caption{コネクタ形状(テキストp.107から引用)}
      \end{figure}
    \end{minipage}

\end{frame}




\begin{frame}{References}

  \begin{thebibliography}{99}
    \beamertemplatetextbibitems
    \bibitem{jpnic:ip枯渇} \texttt{https://www.nic.ad.jp/ja/ip/ipv4pool/}
    \bibitem{wimax:napt} \texttt{https://www.uqwimax.jp/wimax/plan/option/global\_ip/}
    \bibitem{ninten:sw} \texttt{https://www.nintendo.co.jp/support/switch/internet/wired.html}
    \bibitem{ninten:gc} \texttt{https://www.nintendo.co.jp/ngc/acce/gc\_bba.pdf}
    \bibitem{jeitacat8} \texttt{https://home.jeita.or.jp/upload\_file/20180618105908\_H95U0A2VLc.pdf}
  \end{thebibliography}

\end{frame}

\end{document}
