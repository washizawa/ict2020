\documentclass[12pt,aspectratio=169]{beamer}
\usetheme{default}
\usecolortheme{dolphin}
\usefonttheme{structurebold}
\setbeamertemplate{footline}[frame number]

\title{ShellScript 03}
\author{@aoirint}
% \institute{}
% \author{@aoirint}
\date{2020/05/14}

\begin{document}

% 01
\frame{\maketitle}

% 02
\begin{frame}{テキスト}

  \begin{minipage}{0.58\textwidth}
    \begin{itemize}
      \item 新しいシェルプログラミングの教科書
      \begin{itemize}
        \item 著・三宅英明
        \item 刊・SB Creative
      \end{itemize}
    \end{itemize}
  \end{minipage}
  \hfill
  \begin{minipage}{0.38\textwidth}
    \vspace{-4\baselineskip}
    \begin{center}
      \includegraphics[width=5cm,bb=0 0 467 596]{../01/images/shellbook.jpg}
    \end{center}
  \end{minipage}

  \begin{itemize}
    \item 書影
    \begin{itemize}
      \item { \small \url{https://www.sbcr.jp/product/4797393101/} }
    \end{itemize}
  \end{itemize}

\end{frame}


\begin{frame}{今回の内容}

  \begin{itemize}
    \item 第5章 展開とクォーティング
  \end{itemize}

\end{frame}


\begin{frame}{展開とクォーティング}

  \begin{itemize}
    \item 展開
    \begin{itemize}
      \item 特別な記号(メタ文字、*や\{\}など)を他の文字列に置き換える機能
      \item まとめて処理するファイルを指定する
      \item 規則的な値を変数に代入する
      \item 変数に操作を加えてコマンドに渡す
    \end{itemize}

    \item クォーティング
    \begin{itemize}
      \item 展開してほしくないときに抑制する(エスケープする)こと
    \end{itemize}

  \end{itemize}

\end{frame}


\begin{frame}{今回の内容}

  \begin{itemize}
    \item 第5章 展開とクォーティング
    \begin{itemize}
      \item パス名展開(改めて)
      \item ブレース展開
      \item チルダ展開
      \item パラメータ展開(多機能)
      \item ---ここまで---
      \begin{scriptsize}
        \item 算術式評価
        \item 算術式展開
        \item プロセス置換
        \item 履歴展開
        \item クォーティング
      \end{scriptsize}
    \end{itemize}

  \end{itemize}

\end{frame}


\begin{frame}{パス名展開 1/4}

  \begin{itemize}
    \item ワイルドカード展開、ファイル名グロブ
    \item 条件にマッチするファイル名を並べた文字列として展開する
    \item 条件はディレクトリの区切りを越えない
    \item マッチするファイルが1つもない場合、展開されない

  \end{itemize}

\end{frame}


\begin{frame}{パス名展開 2/4}

  \begin{itemize}
    \item 任意の文字(\texttt{ *, ? })
      \begin{itemize}
        \item *.jpg: photo.jpg, image.jpg, ...
        \item *.txt: document.txt, readme.txt, ...
        \item file.*: file.bin, file.dat, ...
        \item *.*: photo.jpg, readme.txt, ...(拡張子のないファイルにはマッチしない)
        \item iPhone?.txt: iPhoneX.txt, iPhone8.txt, ...
        \item iPhone*.txt: iPhoneX.txt, iPhone11.txt, iPhoneSE.txt, ...
      \end{itemize}

  \end{itemize}

\end{frame}


\begin{frame}{パス名展開 3/4}

  \begin{itemize}
    \item いずれか1文字(\texttt{ [], [!], [\^{}] })
      \begin{itemize}
        \item Android[4-8].txt: Android4.txt, Android5.txt, ..., Android8.txt
        \item Android[!45].txt: ..., Android6.txt, Android7.txt, Android8.txt, ...
        \item Android[\^{}45].txt: !と同じ
      \end{itemize}

  \end{itemize}

\end{frame}


\begin{frame}{パス名展開 4/4}

  \begin{itemize}
    \item 条件はディレクトリの区切りを越えない
      \begin{itemize}
        \item /usr/bin/*sh: /usr/bin/bash, /usr/bin/fish, /usr/bin/zsh, /usr/bin/ssh, ...
        \item /usr/*sh: マッチなし(文字列 /usr/*sh)
        \item /usr/*/*sh: /usr/bin/bash, ..., /usr/lib/openssh, ...
      \end{itemize}

    \item マッチするファイルが1つもない場合、展開されない
      \begin{itemize}
        \item /hogehoge/fuga*piyo: マッチなし(文字列 /hogehoge/fuga*piyo)
      \end{itemize}

  \end{itemize}

\end{frame}


\begin{frame}{ブレース展開 1/3}

  \begin{itemize}
    \item \{, \}:ブレース brace
      \begin{itemize}
        \item\ [, ]:ブラケット bracket
        \item (, ):パレンセシス parenthesis
      \end{itemize}

    \item 文字列の一部を複数のパターンに展開する
      \begin{itemize}
        \item 連番・組み合わせが簡単に作れる
      \end{itemize}
    \item パス名展開と違いファイルの存在とは関係ない

  \end{itemize}

\end{frame}


\begin{frame}{ブレース展開 2/3}

  \begin{itemize}
    \item iPhone \{X,11,SE\}
      \begin{itemize}
        \item iPhone X
        \item iPhone 11
        \item iPhone SE
      \end{itemize}

    \item Android \{6..8\}
      \begin{itemize}
        \item Android 6
        \item Android 7
        \item Android 8
      \end{itemize}

  \end{itemize}

\end{frame}


\begin{frame}{ブレース展開 3/3}

  \begin{itemize}
    \item PHP \{1..9..2\}(1から9まで2ずつ値が増える)
      \begin{itemize}
        \item PHP 1
        \item PHP 3
        \item PHP 5
        \item PHP 7
        \item PHP 9
      \end{itemize}

    \item \{A..Z\}\{A..Z\}(組み合わせ)
      \begin{itemize}
        \item AA
        \item AB
        \item ...
        \item ZY
        \item ZZ
      \end{itemize}
    \item 日本語はだめそう(\{あ..お\}はbashで動かなかった)

  \end{itemize}

\end{frame}


\begin{frame}{チルダ展開}

  \begin{itemize}
    \item \~{}:チルダ
    \item ホームディレクトリを表す
      \begin{itemize}
        \item ホームディレクトリ:ふつう、/home/USERNAMEのこと
      \end{itemize}
    \item \~{}:自分のホームディレクトリ
      \begin{itemize}
        \item \~{}/.bashrc:bashの設定ファイル(ログイン時自動実行スクリプト)
      \end{itemize}
    \item \~{}someone:ユーザsomeoneのホームディレクトリ

  \end{itemize}

\end{frame}


\begin{frame}{パラメータ展開 1/12:基本}

  \begin{itemize}
    \item パラメータ:変数のこと
    \item 変数の値に置き換える
    \item ちょっと値に操作を加えることもできる
      \begin{itemize}
        \item できる操作が多いのでこの節は長め
      \end{itemize}
  \end{itemize}

\end{frame}


\begin{frame}{パラメータ展開 2/12:シンプル}

  \begin{itemize}
    \item \$VAR:変数VARの値
    \item \$\{VAR\}:上と同じ
  \end{itemize}

\end{frame}


\begin{frame}{パラメータ展開 3/12:デフォルト値}

  \begin{itemize}
    \item デフォルト値の指定(:-)
      \begin{itemize}
        \item 変数に値が設定されていない(空文字列を含む)ときのデフォルト値を指定する
        \item \$\{VAR:-myvalue\}
          \begin{itemize}
            \item VARに値が設定されているときはVARの値
            \item 設定されていないときはmyvalue
            \item 擬似コード:isset(VAR) ? VAR : 'myvalue'
          \end{itemize}
      \end{itemize}
    \item デフォルト値の指定(-)
      \begin{itemize}
        \item 変数に値が設定されていない(空文字列を含まない)ときのデフォルト値を指定する
        \item \$\{VAR-myvalue\}
      \end{itemize}
    \item 位置パラメータ(引数を取り出す変数)とよく組み合わせられる
      \begin{itemize}
          \item \$\{1:-\$HOME/.conf\}:引数1が渡されていないとき\$HOME/.conf
      \end{itemize}

  \end{itemize}

\end{frame}


\begin{frame}{パラメータ展開 4/12:デフォルト値2}

  \begin{itemize}
    \item デフォルト値を代入しつつ展開(:=)
      \begin{itemize}
        \item 変数に値が設定されていない(空文字列を含む)とき、変数に値を設定しつつ展開する
        \item \$\{VAR:=myvalue\}
          \begin{itemize}
            \item VARに値が設定されているときはVARの値(代入はしない)
            \item 設定されていないときはVARにmyvalueを代入した上でmyvalue
          \end{itemize}
      \end{itemize}

  \end{itemize}

\end{frame}


\begin{frame}{パラメータ展開 5/12:値の検証}

  \begin{itemize}
    \item 値の検証(:?)
      \begin{itemize}
        \item 変数に値が設定されていない(空文字列を含む)とき、エラーメッセージを出力し、スクリプトを終了する
        \item \$\{VAR:?mymessage\}
          \begin{itemize}
            \item VARに値が設定されているときはVARの値
            \item 設定されていないときは標準エラー出力にmymessageを出力し、スクリプトを終了
          \end{itemize}
      \end{itemize}

  \end{itemize}

\end{frame}


\begin{frame}{パラメータ展開 6/12:値の正規化}

  \begin{itemize}
    \item 値の正規化(:+)
      \begin{itemize}
        \item 変数に値が設定されている(空文字列を含まない)とき、値に置き換える
        \item 設定されていないときは空文字列に置き換える
        \item \$\{VAR:+myvalue\}
          \begin{itemize}
            \item VARに値が設定されているときはmyvalue
            \item 設定されていないときは空文字列
          \end{itemize}
        \item \$\{VAR+myvalue\}も\$\{VAR-myvalue\}と同様
        \item 擬似コード:isset(VAR) ? 'myvalue' : ''
      \end{itemize}

  \end{itemize}

\end{frame}


\begin{frame}{パラメータ展開 7/12:文字列の切り出し}

  \begin{itemize}
    \item 文字列の切り出し(substring)
      \begin{itemize}
        \item 文字列の一部を切り出す
        \item \$\{VAR:start\}, \$\{VAR:start:length\}(start, lengthは数値)
        \item VAR=abcdef, \$\{VAR:2\}
          \begin{itemize}
            \item cdef
          \end{itemize}
        \item VAR=abcdef, \$\{VAR: -2\}(要スペース)
          \begin{itemize}
            \item ef
          \end{itemize}
        \item VAR=abcdef, \$\{VAR:2:2\}
          \begin{itemize}
            \item cd
          \end{itemize}
        \item VAR=abcdef, \$\{VAR:2:-1\}
          \begin{itemize}
            \item cde
          \end{itemize}
      \end{itemize}

  \end{itemize}

\end{frame}


\begin{frame}{パラメータ展開 8/12:配列の切り出し}

  \begin{itemize}
    \item 配列:ARRAY=(a b c d e)で設定できる
    \begin{itemize}
      \item \$\{ARRAY[@]\}:配列として展開
        \begin{itemize}
          \item for elm in "\$\{ARRAY[@]\}"; do echo \$elm; done:出力は要素の個数行
        \end{itemize}
      \item \$\{ARRAY[*]\}:文字列として結合してから展開
        \begin{itemize}
          \item for elm in "\$\{ARRAY[*]\}"; do echo \$elm; done:出力は1行
        \end{itemize}
    \end{itemize}

    \item 配列の切り出し
      \begin{itemize}
        \item 配列の一部を切り出す
        \item \$\{ARRAY[@]:start\}, \$\{ARRAY[@]:start:length\}(start, lengthは数値)
        \item 文字列と同じ
      \end{itemize}

  \end{itemize}

\end{frame}


\begin{frame}{パラメータ展開 8/12:長さ}

  \begin{itemize}
    \item 文字列や配列の長さ(数値)に展開
    \item \$\{\#VAR\}
    \item \$\{\#ARRAY[@]\}, \$\{\#ARRAY[*]\}(どちらも同じ)
  \end{itemize}

\end{frame}




\begin{frame}{パラメータ展開 9/12:パターンマッチ 1/2}

  \begin{itemize}
    \item 変数の値のうちパターン(パス名展開と共通規格)に一致する部分を取り除いて展開
    \item パターンにマッチする部分がない場合はどこも取り除かれない(そのままの値)
    \item 最短前方一致:\$\{VAR\#pattern\}
      \begin{itemize}
        \item 変数の値, パターン:展開後
        \item VAR=hoge.tar.gz, \$\{VAR\#hoge\}:.tar.gz
        \item VAR=hoge.tar.gz, \$\{VAR\#*.\}:tar.gz
        \item VAR=hoge.tar.gz, \$\{VAR\#XYZ\}:hoge.tar.gz
      \end{itemize}
    \item 最長前方一致:\$\{VAR\#\#pattern\}
      \begin{itemize}
        \item VAR=hoge.tar.gz, \$\{VAR\#\#*.\}:gz
      \end{itemize}

  \end{itemize}

\end{frame}

\begin{frame}{パラメータ展開 10/12:パターンマッチ 2/2}

  \begin{itemize}
    \item 最短後方一致:\$\{VAR\%pattern\}
      \begin{itemize}
        \item 変数の値, パターン:展開後
        \item VAR=hoge.tar.gz, \$\{VAR\%.*\}:hoge.tar
      \end{itemize}
    \item 最長後方一致:\$\{VAR\%\%pattern\}
      \begin{itemize}
        \item VAR=hoge.tar.gz, \$\{VAR\%\%.*\}:hoge
      \end{itemize}

    \item 例
      \begin{itemize}
        \item ファイル名を取り出す(basename):\$\{VAR\#\#*/\}
          \begin{itemize}
            \item /usr/bin/bash:bash
          \end{itemize}
        \item 親ディレクトリまでのパスを取り出す(dirname):\$\{VAR\%/*\}
          \begin{itemize}
            \item /usr/bin/bash:/usr/bin
          \end{itemize}

      \end{itemize}

  \end{itemize}

\end{frame}


\begin{frame}{パラメータ展開 11/12:パターン置換}

  \begin{itemize}
    \item 最初の1つを置換:\$\{VAR/pattern/replacement\}
      \begin{itemize}
        \item 変数の値, パターン:展開後
        \item VAR=hoge.inc.php, \$\{VAR/./\_\}:hoge\_inc.php
        \item VAR=hoge.inc.php, \$\{VAR/*./fuga.\}:fuga.php
      \end{itemize}
    \item すべて置換:\$\{VAR//pattern/replacement\}
      \begin{itemize}
        \item VAR=hoge.inc.php, \$\{VAR//./\_\}:hoge\_inc\_php
      \end{itemize}

  \end{itemize}

\end{frame}


\begin{frame}{パラメータ展開 12/12:配列とパターン}

  \begin{itemize}
    \item 変数部に配列を指定した場合、配列の各要素についてそれぞれ除去、置換される
    \item ARRAY=(hoge.tar.gz fuga.tar.xz piyo.tar.bz2)
    \item \$\{ARRAY[@]\%\%.*\}:hoge fuga piyo
    \item \$\{ARRAY[@]/./\_\}:hoge\_tar.gz fuga\_tar.xz piyo\_tar.bz2
    \item \$\{ARRAY[@]//./\_\}:hoge\_tar\_gz fuga\_tar\_xz piyo\_tar\_bz2

  \end{itemize}

\end{frame}


\end{document}
