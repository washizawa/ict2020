\documentclass[12pt,aspectratio=169]{beamer}
\usetheme{default}
\usecolortheme{dolphin}
\usefonttheme{structurebold}
\setbeamertemplate{footline}[frame number]

\title{ShellScript 06}
\author{@aoirint}
\date{2020/07/09}

\begin{document}

% 01
\frame{\maketitle}

% 02
\begin{frame}{テキスト}

  \begin{minipage}{0.58\textwidth}
    \begin{itemize}
      \item 新しいシェルプログラミングの教科書
      \begin{itemize}
        \item 著・三宅英明
        \item 刊・SB Creative
      \end{itemize}
    \end{itemize}
  \end{minipage}
  \hfill
  \begin{minipage}{0.38\textwidth}
    \vspace{-4\baselineskip}
    \begin{center}
      \includegraphics[width=5cm,bb=0 0 467 596]{../01/images/shellbook.jpg}
    \end{center}
  \end{minipage}

  \begin{itemize}
    \item 書影
    \begin{itemize}
      \item { \small \url{https://www.sbcr.jp/product/4797393101/} }
    \end{itemize}
  \end{itemize}

\end{frame}


\begin{frame}{前回の内容}

  \begin{itemize}
    \item 第5章 展開とクォーティング
    \item 第6章 制御構造
    \begin{itemize}
      \item if
      \item testコマンド([、多機能)
      \item 文字列の比較
    \end{itemize}
  \end{itemize}

\end{frame}

\begin{frame}{今回の内容}

  \begin{itemize}
    \item 第6章 制御構造
    \begin{itemize}
      \item 整数の比較
      \item ファイル属性の比較
      \item if文の注意事項
      \item \&\&と\textbar\textbar
      \item \lbrack\lbrack\ \rbrack\rbrack
      \item for
      \item case
      \item while, until
    \end{itemize}

  \end{itemize}

\end{frame}

\begin{frame}{if、testコマンド(条件分岐)}

  \begin{itemize}
    \item {[ "\$1" = 'yes' ] \# 終了ステータスに結果}
    \item
      {if [ "\$1" = 'yes' ]; then \\
        \hspace{0.5cm}echo YES \\
      else \\
        \hspace{0.5cm}echo NO \\
      fi}
    \vspace{0.5cm}
    \item \lbrack はtestコマンド(Unixコマンドの1つ)
      \begin{itemize}
        \item 真ならば終了ステータス0
        \item 偽ならば終了ステータス1
      \end{itemize}
    \item ifなどに使う"条件"には任意のコマンドが使える(コマンドが成功=終了ステータス0ならば、という条件)
    % cdコマンド、存在しないファイル $?
  \end{itemize}
\end{frame}

\begin{frame}{整数の比較}

  \begin{itemize}
    \item 整数変数の定義
      \begin{itemize}
        \item declare -i VAR
        \item VAR=123
      \end{itemize}

    \item int1 -eq int2:等しい
    \item int1 -ne int2 :等しくない
    \item int1 -lt int2 :int1がint2より小さい
    \item int1 -le int2 :int1がint2以下
    \item int1 -gt int2 :int1がint2より大きい
    \item int1 -ge int2 :int1がint2以上
    \vspace{0.5cm}
    \item $<$や$>$は文字列用の演算子(辞書順比較)
  \end{itemize}

\end{frame}

\begin{frame}{ファイル属性の比較 1/2}

  \begin{itemize}
    \item -a file:ファイルが存在する(-eと同じ)
    \item -e file:ファイルが存在する(-aと同じ)
    \item -f file:ファイルが存在し、通常のファイルである
    \item -d file:ファイルが存在し、ディレクトリである
    \item -h file:ファイルが存在し、シンボリックリンクである
    \item file1 -nt file2:file1の更新時刻がfile2より新しい
    \item file1 -ot file2:file1の更新時刻がfile2より古い
  \end{itemize}

\end{frame}

\begin{frame}{ファイル属性の比較 2/2}

  \begin{itemize}
    \item -G file:ファイルが存在し、グループが現在実行中のシェルのグループ
    \item -O file:ファイルが存在し、オーナーが現在実行中のシェルのユーザ
    \item -r file:ファイルが存在し、読み取り権限が与えられている
    \item -w file:ファイルが存在し、書き込み権限が与えられている
    \item -x file:ファイルが存在し、実行権限が与えられている
  \end{itemize}

\end{frame}

\begin{frame}{演算子の結合}

  \begin{itemize}
    \item 条件式1 -a 条件式2:AND
    \item 条件式1 -o 条件式2:OR
    \item ! 条件式:NOT
    \item ():条件式のグループ化
  \end{itemize}

\end{frame}

\begin{frame}{if文の注意事項:ifと条件部のスペース}

  \begin{itemize}
    \item ifと条件部、条件部の中身はそれぞれスペースで区切る
    \item ifと\lbrack はそれぞれ別のコマンド
    \begin{itemize}
      \item ifは直後のコマンドを実行、終了ステータスで条件分岐するコマンド
      \item \lbrack は条件式を評価して終了ステータスで結果を返すコマンド
    \end{itemize}
    \item \lbrack の条件部は引数
    \begin{itemize}
      \item 最後の引数は\rbrack と決まっている
    \end{itemize}
    \item ダメな例
    \begin{itemize}
      \item if\lbrack "\$str" = NG\rbrack ; then echo "NG"; fi
    \end{itemize}
    \item OKな例
    \begin{itemize}
      \item if \lbrack\ "\$str" = OK \rbrack ; then echo "OK"; fi
      \item ifと\lbrack の間にスペース
      \item \lbrack の後、\rbrack の前にスペース
    \end{itemize}

  \end{itemize}

\end{frame}

\begin{frame}{if文の注意事項:条件部の後ろ}

  \begin{itemize}
    \item 条件部の後ろは;で区切る
    \item ダメな例
    \begin{itemize}
      \item if \lbrack "\$str" = NG\rbrack\ then echo "NG"; fi
    \end{itemize}
    \item OKな例
    \begin{itemize}
      \item if \lbrack "\$str" = OK\rbrack ; then echo "OK"; fi
      \item \rbrack とthenの間に;
    \end{itemize}

  \end{itemize}

\end{frame}

\begin{frame}{if文の注意事項:記号のクォーティング}

  \begin{itemize}
    \item \lbrack コマンドの内部でクォーティングが必要
    \begin{itemize}
      \item $<$、$>$:リダイレクト
      \item (、):サブシェル
    \end{itemize}
    \item ダメな例
    \begin{itemize}
      \item if \lbrack "\$str1" $<$ "\$str2"\rbrack ; then echo "NG"; fi
      \item if \lbrack\ ( "\$str1" $<$ "\$str2" ) \rbrack ; then echo "NG"; fi
    \end{itemize}
    \item OKな例
    \begin{itemize}
      \item if \lbrack "\$str1" \textbackslash$<$ "\$str2"\rbrack ; then echo "OK"; fi
      \item if \lbrack "\$str1" '$<$' "\$str2"\rbrack ; then echo "OK"; fi
      \item if \lbrack "\$str1" "$<$" "\$str2"\rbrack ; then echo "OK"; fi
      \item if \lbrack\ \textbackslash( "\$str1" '$<$' "\$str2" \textbackslash) \rbrack ; then echo "OK"; fi
    \end{itemize}

  \end{itemize}

\end{frame}

\begin{frame}{\&\&と\textbar\textbar}

  \begin{itemize}
    \item 複数のコマンドをAND、ORで連結して実行する記号
    \item cmd1 \&\& cmd2:AND
    \begin{itemize}
      \item cmd1の終了ステータスが0のときにcmd2を実行する
    \end{itemize}
    \item cmd1 \textbar\textbar\ cmd2:OR
    \begin{itemize}
      \item cmd1の終了ステータスが0"以外"のときにcmd2を実行する
    \end{itemize}

  \end{itemize}

\end{frame}

\begin{frame}{testコマンドと\&\&と\textbar\textbar}

  \begin{itemize}
    \item ifに与える条件となるコマンドを複数にできる
    \item if \lbrack\ -n "\$file" \rbrack\ \&\& \lbrack\ ! -e "\$file" \rbrack; then touch "\$file"; fi
    \begin{itemize}
      \item -n:ファイル名が空文字列でない
      \item !:NOT
      \item -e:ファイルが存在する
    \end{itemize}

  \end{itemize}

\end{frame}

\begin{frame}{\lbrack\lbrack\ \rbrack\rbrack}

  \begin{itemize}
    \item 基本的には\lbrack と同じ
    \item \lbrack と違いコマンドではなくbashの構文
    \begin{itemize}
      \item コマンドでは\&\&や\textbar\textbar、$<>$や()を特別扱いできない
    \end{itemize}
    \item シンプルに条件式を書ける
    \item -aや-oの代わりに\&\&や\textbar\textbar を使える
    \begin{itemize}
      \item if \lbrack\lbrack\ \$x -gt 3 \&\& \$x -lt 7 \rbrack\rbrack ; then echo "3 $<$ x $<$ 7"; fi
    \end{itemize}
    \item 単語分割とパス名展開が行われない
    \item $<>$や()をクォートする必要がない
  \end{itemize}

\end{frame}

\begin{frame}{\lbrack\lbrack\ \rbrack\rbrack と単語分割}

  \begin{itemize}
    \item 単語分割
    \begin{itemize}
      \item パラメータ展開、コマンド置換、算術式展開を行った後の文字列をスペース、タブ、改行によって複数の単語に区切る機能
      \item 例)スペースを含む文字列をパラメータ展開
    \end{itemize}
    \item str='abc xyz'
    \item if \lbrack\ \$str = 'abc xyz' \rbrack ; then echo "NG"; fi
    \begin{itemize}
      \item if \lbrack\ 'abc' 'xyz' = 'abc xyz' \rbrack ; then echo "NG"; fi
    \end{itemize}
    \item クォートによる回避
    \begin{itemize}
      \item if \lbrack\ "\$str" = 'abc xyz' \rbrack ; then echo "OK"; fi
    \end{itemize}
    \item \lbrack\lbrack\ \rbrack\rbrack による回避
    \begin{itemize}
      \item if \lbrack\lbrack\ \$str = 'abc xyz' \rbrack\rbrack ; then echo "OK"; fi
    \end{itemize}
  \end{itemize}

\end{frame}

\begin{frame}{\lbrack\lbrack\ \rbrack\rbrack とパターンマッチ}

  \begin{itemize}
    \item パターン文字列
    \begin{itemize}
      \item パス名展開に使われる文字列
      \item 例)*.jpg
    \end{itemize}
    \item \lbrack\lbrack\ \rbrack\rbrack の内側では通常パス名展開されない
    \item パターンマッチ
    \begin{itemize}
      \item ==、!=の右辺にパス名展開の記号がある場合(変数の内容の場合を含む)
      \item str=abc
      \item pattern=a*
      \item if \lbrack\lbrack\ \$str == \$pattern \rbrack\rbrack ; then echo "OK"; fi \# aから始まる文字列で真
      \item if \lbrack\lbrack\ \$str == "\$pattern" \rbrack\rbrack ; then echo "NG"; fi \# a*という文字列で真
    \end{itemize}
  \end{itemize}

\end{frame}

\begin{frame}{\lbrack\lbrack\ \rbrack\rbrack とパターンマッチ}

  \begin{itemize}
    \item 拡張正規表現による比較
    \begin{itemize}
      \item =~:右辺に拡張正規表現
      \item str=abc
      \item pattern=a*
      \item if \lbrack\lbrack\ \$str =~ \$pattern \rbrack\rbrack ; then echo "OK"; fi \# aから始まる文字列で真
    \end{itemize}
  \end{itemize}

\end{frame}

\begin{frame}{for}

  \begin{itemize}
    \item 単語リストに対して繰り返し処理する
    \item for item in aaa bbb ccc; do echo "\$item"; done
    \vspace{0.5cm}
    \item words='aaa bbb ccc'
    \item for item in \$words; do echo "\$item"; done
    \vspace{0.5cm}
    \item for file in *.txt; do echo "\$file"; done
    \vspace{0.5cm}
    \item for arg in "\$@"; do echo "\$arg"; done
    \begin{itemize}
      \item シェルスクリプトの引数リスト
    \end{itemize}
  \end{itemize}

\end{frame}

\begin{frame}{for:break,continue}

  \begin{itemize}
    \item break
    \begin{itemize}
      \item 繰り返しの中断
    \end{itemize}
    \item continue
    \begin{itemize}
      \item 次の繰り返しへ
    \end{itemize}
    \item for item in aaa bbb ccc \\ do \\ \hspace{0.5cm} if [[ "\$item" == 'bbb' ]] \\\hspace{0.5cm}  then \\ \hspace{1.0cm} continue \\ \hspace{0.5cm} fi \\ \hspace{0.5cm} echo "\$item" \\ done
  \end{itemize}

\end{frame}

\begin{frame}{case}

  \begin{itemize}
    \item 1つの文字列に対して複数のパターンを指定
    \item 最初にマッチしたパターンに対応する処理を実行する
    \item \textbar :OR
    \item case 文字列 in \\
    \hspace{0.5cm} パターン1) \\
    \hspace{1.0cm} 処理1 \\
    \hspace{1.0cm} ;; \\
    \hspace{0.5cm} パターン2 \textbar パターン3) \\
    \hspace{1.0cm} 処理2 \\
    \hspace{1.0cm} ;; \\
    esac
  \end{itemize}

\end{frame}

\begin{frame}{case}

  \begin{itemize}
    \item 1つの文字列に対して複数のパターンを指定
    \item 最初にマッチしたパターンに対応する処理を実行する
    \item case "\$file" in \\
    \hspace{0.5cm} *.zip) \\
    \hspace{1.0cm} unzip "\$file" \\
    \hspace{1.0cm} ;; \\
    \hspace{0.5cm} *.tar.gz \textbar\ *.tgz) \\
    \hspace{1.0cm} tar xzf "\$file" \\
    \hspace{1.0cm} ;; \\
    esac
  \end{itemize}

\end{frame}

\begin{frame}{whileとuntil}

  \begin{itemize}
    \item while 条件コマンド; do 実行コマンド; done\\
    \begin{itemize}
      \item 条件コマンドの終了ステータスが0である限り繰り返し
    \end{itemize}
    \item until 条件コマンド; do 実行コマンド; done\\
    \begin{itemize}
      \item 条件コマンドの終了ステータスが0以外である限り繰り返し
    \end{itemize}
    \item \lbrack, \lbrack\lbrack\ \rbrack\rbrack, break, continueも使える
  \end{itemize}

\end{frame}


\end{document}
